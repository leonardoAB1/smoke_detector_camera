\begin{table}[!ht]
\begin{adjustbox}{width=\columnwidth,center}
\centering
\small
    \begin{tabularx}{\textwidth}{|l|X|l|l|l|l|l|l|l|l|l|}
    \hline
        ~ & Objet. & Act. & Inicial & Esc1 & Esc2 & Esc3 & Esc4 & Esc5 & Esc6\\ \hline
        Diám(mm) & ~ & 5 & 2 & 2.50 & 3 & 3.50 & 4 & 4.50 & 5 \\ \hline
        Tensión($N/m^2$) & $10^7$ & 3430 & 8 & 8 & 7 & 8 & 1 & 7 & 9\\ \hline
        Masa($g$) & Min& 167.2 & 163.4 & 164.0 & 164.6 & 165.3 & 165.9 & 166.5 & 167.2\\ \hline
    \end{tabularx}
    \end{adjustbox}
    \label{tab:estudioDisenoBase}
    \caption{Resultados de Estudio de Diseño para la Base de la Cámara. Se puede notar en verde los puntos de mayor tensión, en rojo la gravedad y en rosado la carga distribuida de la cámara en la estructura.}
\end{table}

En base al análisis previo se realizó una simulación de carga estática especifica de la base de la cámara a fin de optimizar el diámetro de los pernos que sostienen la estructura como se aprecia en la Figura~\ref{fig:simBase}.
El Cuadro~\ref{tab:estudioDisenoBase} expone los resultados obtenidos en el estudio. En el mismo se determinó que el diámetro óptimo es aquel que utiliza el mínimo permitido por el método de manufactura, es decir, 2 mm. 