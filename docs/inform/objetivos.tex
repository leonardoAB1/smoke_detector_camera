El proyecto "Smoke Detector Cam" tiene los siguientes objetivos:

\subsection{Objetivo General}
Desarrollar el mecanismo de movimiento del sistema embebido "Smoke Detector Cam" para la detección temprana de humo y fuego en ambientes cerrados, con el propósito de contribuir a la prevención y mitigación de incendios y promover ambientes libres de humo de tabaco en cumplimiento de la ley de control del tabaco en Bolivia.

\needspace{3cm}
\subsection{Objetivos Específicos}
\begin{itemize}
\item Desarrollar "Smoke Detector Cam" con sensores de humo, cámara, refrigeración y batería, teniendo en cuenta limitaciones de espacio y fabricación.
\item Crear el cuerpo de la cámara con software CAD para impresión 3D a gran escala, enfocándose en resistencia, durabilidad y adaptabilidad a diferentes entornos cerrados.
\item Simular el sistema de refrigeración planificado para garantizar el correcto funcionamiento de los componentes electrónicos sin afectar la apariencia visual, utilizando herramientas de análisis de ingeniería asistida por computadora (CAE).
\needspace{4cm}
\item Aplicar el método de gestión de proyectos de cinco pasos de la literatura académica (\cite{Ulrich2012}), que incluye identificar oportunidades, evaluar proyectos, asignar recursos y tiempo, finalizar la planificación del anteproyecto y reflexionar sobre los resultados y el proceso.
\end{itemize}