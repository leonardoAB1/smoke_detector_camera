El tabaquismo es una epidemia mundial con graves consecuencias para la salud, la sociedad y la economía. En Bolivia, aproximadamente el 21,9\% de los hombres y alrededor del 9\% de las mujeres consumen tabaco diariamente(\cite{BoliviaTobaccoControlLaw}). Además, un preocupante 46,6\% de los jóvenes están expuestos al humo de tabaco ajeno, y cada año más de 4.600 bolivianos y bolivianas pierden la vida debido a enfermedades relacionadas con el consumo de tabaco.

Para abordar esta preocupante situación, se promulgó una importante ley de control del tabaco en Bolivia. Misma que estableció que todos los espacios cerrados de acceso público y lugares de trabajo deben ser 100\% libres de humo de tabaco, protegiendo así a toda la población de los riesgos asociados a la exposición al humo de segunda mano. Según lo reportado por la Organización Mundial de la Salud (OMS) se estima que alrededor de 8 millones de personas mueren cada año debido al tabaquismo, de las cuales 7 millones de muertes son atribuidas al consumo directo de tabaco y 1 millón de muertes son causadas por el humo de tabaco ajeno.

Además de los impactos en la salud, el tabaquismo también genera una significativa carga económica, con la OMS estimando que la economía mundial tiene que soportar más de USD 500 mil millones cada año debido a problemas relacionados con el tabaquismo.

La vigilancia basada en inteligencia artificial de las áreas de no fumar para detectar fumadores como posibles infractores es esencial(\cite{Khan2022b}). Al tener una solución automatizada y precisa, las autoridades pueden aplicar medidas correctivas oportunas y fomentar el cumplimiento de las regulaciones, mejorando así la calidad de vida y el bienestar general en el contexto de ciudades inteligentes.
Esta solución puede contribuir significativamente al monitoreo y prevención del consumo de tabaco en espacios donde está prohibido fumar.