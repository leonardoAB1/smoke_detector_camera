En el proceso de desarrollo del sistema "Smoke Detector Cam" para la detección temprana de humo y fuego en ambientes cerrados, se han tomado decisiones clave, como la elección de motores DC, el uso de un mecanismo de tren de engranajes diferencial, baterías de Li-Ion, abrazaderas para la gestión de cables y un sistema de refrigeración con un ventilador ubicado sobre las placas electrónicas. Sin embargo, es esencial considerar ciertos aspectos para garantizar el éxito y la eficacia del proyecto.

En primer lugar, la pieza o sección que requiere mayor consideración es el diseño de la carcasa. El diseño actual no permite proteger al mecanismo en todos su circunferencia de partículas de polvo o humo así como también depende en gran medida del ajuste por apriete para mantenerse en posición. En su defecto de pegamentos, restringiendo los grados de libertad de la cámara en este caso.

En cuanto al mecanismo de tren de engranajes diferencial, se tiene pensado realizar pruebas rigurosas para asegurar su robustez y eficiencia, garantizando su capacidad de funcionar en condiciones variables de humedad y temperatura.

Queda pendiente además de las abrazaderas para cables, la implementación de conectores o interfaces que faciliten el ensamblaje y el mantenimiento del sistema.

Por otro lado, las pruebas de flujo de aire y simulaciones de temperatura son necesarias para garantizar que el ventilador sea capaz de mantener las temperaturas dentro de rangos seguros. La instalación de sensores de temperatura para un monitorio activo puede ser una medida adicional para mantener un control preciso de la temperatura.

Más aun, se debe revisar cada una de las piezas diseñadas para optimizarlas mismas para su manufactura mediante impresión 3D. Por ejemplo, es necesario tomar en cuenta que no se recomiendan agujeros de diámetro menor o igual a dos milímetros y mas aun, independientemente del diámetro si no se toman consideraciones para permitir la flexibilidad de los mismos resulta difícil asegurar el ajuste por apriete en la producción en masa de piezas impresas.

Finalmente, las pruebas y validaciones en diferentes entornos y condiciones son esenciales para garantizar el rendimiento del "Smoke Detector Cam". Además, considerar la implementación de un sistema de registro de datos para recopilar información sobre el funcionamiento del dispositivo puede ser valioso para análisis posteriores y mejoras en el diseño.