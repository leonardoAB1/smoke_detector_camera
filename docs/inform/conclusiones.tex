En conclusión, el proyecto de desarrollo del sistema embebido 'Smoke Detector Cam', concebido para detectar tempranamente humo y fuego en ambientes cerrados, fue delineado eficazmente a través de una serie de objetivos específicos. Desde su primera versión, enfocada en formalizar la estructura y la mecánica de movimiento de la cámara, hasta las iteraciones posteriores, el proyecto experimentó notables evoluciones.

La segunda versión priorizó la mejora estructural para una instalación rápida y pruebas en tiempo real, integrando más sensores y LED infrarrojos para un funcionamiento óptimo. Esta versión consideró módulos de alimentación para garantizar la circulación adecuada de corriente. En la tercera iteración, se implementó un método de gestión de proyectos de cinco pasos, redefiniendo por completo el diseño de la cámara. Se mejoró el movimiento, la refrigeración y la disposición de los componentes electrónicos para un montaje vertical sobre superficies planas y una visión más amplia.

La última versión, la cuarta, incorporó un lente protector para salvaguardar la cámara y mejoró las piezas mecánicas para una refrigeración más eficiente. Cada versión fue trabajada con software CAD, optimizando el diseño para la impresión 3D, asegurando resistencia, durabilidad y adaptabilidad a distintos entornos cerrados.

Este enfoque buscaba contribuir a la prevención y mitigación de incendios, así como fomentar ambientes libres de humo de tabaco, en estricta conformidad con la ley de control del tabaco en Bolivia. En resumen, el proyecto 'Smoke Detector Cam' fue rediseñado múltiples veces, garantizando una metodología sólida y estructurada. Este enfoque prometía resultados eficaces en la detección de humo y fuego en entornos cerrados, con implicaciones significativas para la seguridad y el cumplimiento de regulaciones.