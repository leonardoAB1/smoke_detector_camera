\calwidth=15.5cm

% Few useful commands (our classes always meet either on Monday and Wednesday 
% or on Tuesday and Thursday)

\newcommand{\MWClass}{%
\calday[Monday]{\classday}
\calday[Tuesday]{\classday}
\calday[Wednesday]{\classday} 
\calday[Thursday]{\classday}  
\calday[Friday]{\classday}
\skipday\skipday % weekend (no class)
}

\newcommand{\TRClass}{%
\calday[Monday]{\classday}
\calday[Tuesday]{\classday} 
\calday[Wednesday]{\classday}
\calday[Thursday]{\classday} 
\calday[Friday]{\classday} % 
\skipday\skipday % weekend (no class)
}

\newcommand{\Holiday}[2]{%
\options{#1}{\noclassday}
\caltext{#1}{#2}
}

%\paragraph*{Calendario del Proyecto Smoke Detector Cam:}
\begin{center}
\begin{calendar}{08/28/2023}{15} % Semester starts on 1/11/2010 and last for 18

% weeks, including finals week
\setlength{\calboxdepth}{.25in}
\TRClass
% schedule

% Semana 5
\caltexton{0}{}

% Semana 6
\caltexton{6}{}

% Semana 7
\caltexton{5}{Presentación de propuestas de proyecto\cellcolor{green}}
\caltexton{7}{Objetivo 1.1}
\caltexton{10}{Objetivo 1.2}

% Semana 8
\caltexton{12}{}
\caltexton{15}{Metodo de los 5 pasos\cellcolor{green}}


% Semana 9 Examenes
\caltexton{16}{\cellcolor{red}}
\caltextnext{Objetivo 2.1\cellcolor{red}}
\caltextnext{\cellcolor{red}}
\caltextnext{\cellcolor{red}}
\caltextnext{\cellcolor{red}}
\caltexton{20}{Objetivo 1.4\cellcolor{red}}

% Semana 10 Examenes
\caltexton{21}{\cellcolor{red}}
\caltextnext{Objetivo 1.5\cellcolor{red}}
\caltextnext{\cellcolor{red}}
\caltextnext{\cellcolor{red}}
\caltextnext{\cellcolor{red}}

% Semana 11 Examenes
\caltexton{26}{\cellcolor{red}}
\caltextnext{\cellcolor{red}}
\caltextnext{\cellcolor{red}}
\caltextnext{\cellcolor{red}}
\caltextnext{\cellcolor{red}}

% Semana 12 Examenes
\caltexton{31}{\cellcolor{red}}
\caltextnext{\cellcolor{red}}
\caltextnext{\cellcolor{red}}
\caltextnext{\cellcolor{red}}
\caltextnext{\cellcolor{red}}

% Semana 13
\caltexton{36}{Objetivo 2.1\\Objetivo 2.2}

% Semana 14
\caltexton{41}{Objetivo 2.3}
\caltexton{44}{}

% Semana 15
\caltexton{47}{Objetivo 2.4}
\caltexton{49}{Proyecto Finalizado\cellcolor{green}}

% Semana 16 Examenes
\caltexton{50}{\cellcolor{red}}
\caltextnext{Objetivo 3.1\cellcolor{red}}
\caltextnext{\cellcolor{red}}
\caltextnext{\cellcolor{red}}
\caltextnext{\cellcolor{red}}

% Semana 17 Examenes
\caltexton{55}{\cellcolor{red}}
\caltextnext{Objetivo 3.2\cellcolor{red}}
\caltextnext{\cellcolor{red}}
\caltextnext{\cellcolor{red}}
\caltextnext{\cellcolor{red}}

% Semana 18
\caltexton{60}{}
\caltexton{61}{Objetivo 3.3}

% Semana 19 Finales
\caltexton{65}{\cellcolor{red}}
\caltextnext{\cellcolor{red}}
\caltextnext{\cellcolor{red}}
\caltextnext{\cellcolor{red}}
\caltextnext{\cellcolor{red}}

% Semana 20 Finales
\caltexton{70}{\cellcolor{red}}
\caltextnext{\cellcolor{red}}
\caltextnext{\cellcolor{red}}
\caltextnext{\cellcolor{red}}
\caltextnext{\cellcolor{red}}
\caltext{12/8/2023}{Ultimo día del semestre}

% Feriados
\Holiday{1/1/2023}{Año Nuevo (domingo) \cellcolor{orange}}
\Holiday{1/22/2023}{Estado plurinacional (domingo, festivo especial) \cellcolor{orange}}
\Holiday{2/20/2023}{Carnaval (lunes) \cellcolor{orange}}
\Holiday{2/21/2023}{Carnaval (martes) \cellcolor{orange}}
\Holiday{4/6/2023}{Semana Santa (jueves) \cellcolor{orange}}
\Holiday{4/7/2023}{Semana Santa (viernes) \cellcolor{orange}}
\Holiday{5/1/2023}{Día del Trabajador \cellcolor{orange}}
\Holiday{6/8/2023}{Corpus Christi (jueves) \cellcolor{orange}}
\Holiday{6/21/2023}{Año Nuevo aymara o andino \cellcolor{orange}}
\Holiday{8/7/2023}{Día de la Independencia \cellcolor{orange}}
\Holiday{11/2/2023}{Día de los Difuntos \cellcolor{orange}}
\Holiday{12/25/2023}{Navidad (lunes) \cellcolor{orange}}

\options{4/26/2023}{\noclassday} % finals week
\options{4/27/2023}{\noclassday} % finals week
\options{4/28/2023}{\noclassday} % finals week
\options{4/29/2023}{\noclassday} % finals week
\options{4/30/2023}{\noclassday} % finals week
\caltext{4/27/2023}{\textbf{Final Exam}}
\end{calendar}
\end{center}

\begin{minipage}{\linewidth}
    \subsection*{Objetivo 1: Diseñar el sistema electrónico de la cámara inteligente "Smoke Detector Cam"}

\begin{enumerate}[label=1.\arabic*]
    \item Identificar los componentes esenciales necesarios para el funcionamiento de la cámara inteligente, como sensores de humo, una cámara, sistemas de refrigeración y una batería.
    \item Investigar y seleccionar los componentes electrónicos y mecánicos adecuados que se adapten a las restricciones de espacio y las consideraciones de fabricación.
    \item Diseñar el esquema electrónico que incluya la interconexión de los componentes, considerando las necesidades de alimentación, control y comunicación.
    \item Utilizar software de diseño electrónico, como Eagle, KiCad o Altium, para crear un diseño de circuito impreso (PCB) que acomode los componentes de manera eficiente y cumpla con las restricciones de espacio.
    \item Realizar pruebas de prototipo para asegurarse de que los componentes funcionen correctamente y se comuniquen de manera efectiva.
\end{enumerate}

\subsection*{Objetivo 2: Desarrollar el cuerpo de la cámara utilizando software CAD}

\begin{enumerate}[label=2.\arabic*]
    \item Seleccionar un software CAD adecuado, como SolidWorks, AutoCAD o Fusion 360, para el diseño de la carcasa de la cámara.
    \item Diseñar la carcasa de manera que permita la integración de los componentes electrónicos, como la cámara y los sensores de humo, de manera eficiente tomando en cuenta las limitaciones y predisposiciones del método de manufactura seleccionado.
    \item Modelar el sistema de refrigeración seleccionado utilizando las técnicas de diseño pertinentes para maximizar la comparación y posterior análisis con mínimos recursos.
    \item Ajustar el diseño de la carcasa según sea necesario y realizar revisiones hasta obtener un diseño final satisfactorio.
\end{enumerate}

\subsection*{Objetivo 3: Realizar simulaciones del sistema de refrigeración}
\begin{enumerate}[label=3.\arabic*]
    \item Utilizar herramientas de análisis de ingeniería asistida por computadora (CAE), como ANSYS o COMSOL, para simular el rendimiento del sistema de refrigeración.
    \item Evaluar las temperaturas de funcionamiento de los componentes electrónicos y asegurarse de que estén dentro de los límites seguros.
    \item Optimizar el diseño del sistema de refrigeración según los resultados de las simulaciones.
\end{enumerate}
\end{minipage}